% ==========================
%  LaTeX 2e - Dokument
%  Editor: Dragan Kozulovic
% ==========================
\documentclass[11pt,a4paper,fleqn,twoside]{report}
\usepackage[dvips,final]{graphicx}               %% epsfig -> tex-bilder
\usepackage[latin1]{inputenc}                    %% ISO-Text mit Umlauten
\usepackage[T1]{fontenc}                         %% Zeichensatz mit Umlauten
%\usepackage[lflt]{floatflt}                     %% Textfluss um Bilder
\usepackage[german]{babel}                       %% BABEL -> deutsch
\usepackage{parskip}                             %% parindent=0 / parskip
\usepackage{fancyhdr}                            %% 'fancy'-Ueberschriften
\usepackage[hang,bf]{caption}                    %% Formatierung der captions
\usepackage{color}                               %% Farben
\usepackage{multirow}                            %% Tabellen
\usepackage{natbib}
\usepackage{pdfpages}							 %% PDF ins Dokument integrieren
\usepackage{amsmath}                   			 %% Packages f�r Formeln
\usepackage{amsthm}								 %% -||-
\usepackage{amsfonts}							 %% -||-
\usepackage{amssymb} 							 %% -||-
\usepackage{siunitx}							 %% SI-Einheiten
\usepackage{booktabs}          					 %%table \toprule \bottomrule \midrule
\usepackage{subcaption}
\captionsetup[subfigure]{list=true, font=large, labelfont=bf, 
	labelformat=brace, position=top}
\usepackage[onehalfspacing]{setspace}

% Einzubindende Dateien 
\includeonly{00_frontpage,
             00_empty_page,
	     00_eid_erkl,
	     00_abstract,
	     00_nomenklatur,
	     01_einleitung,
	     02_grundlagen,
	     03_versuchsvorbereitung,
	     04_widerstandsbestimmung,
	     05_windkanalversuche,
	     06_versuchsauswertung,
	     07_fazit,
	     anhang1,
	     anhang2}
\graphicspath{{figures/}}                         %% Pfad fuer einzubindende Graphiken


% Seiten-Layout
\oddsidemargin   0.0cm                           %% Anpassung DIN A4-Format (symmetrisch)
\evensidemargin  0.0cm                           %% Anpassung DIN A4-Format (symmetrisch)
\topmargin      -1.0cm         
\textheight     25.0cm
\textwidth      16.0cm
\pagestyle{fancy}
\renewcommand{\chaptermark}[1]{\markboth{\thechapter.\ #1}{}}
\renewcommand{\sectionmark}[1]{\markright{\thesection\ #1}}
\fancyhf{}                                %Clears all header and footer fields, in preparation.
\fancyhead[LE,RO]{\thepage}               %Displays the page number in bold in the header,
                                          % to the left on even pages and to the right on odd pages.
\fancyhead[RE]{\nouppercase{\leftmark}}   %Displays the upper-level (chapter) information---
                                          % as determined above---in non-upper case in the header, to the right on even pages.
\fancyhead[LO]{\rightmark}                %Displays the lower-level (section) information---as
                                          % determined above---in the header, to the left on odd pages.
\renewcommand{\headrulewidth}{0.5pt}      %Underlines the header. (Set to 0pt if not required).

%\sloppy                                         %% lockerer Zeilenumbruch
%\flushbottom                                    %% buendige letzte Zeile


% Trennungskorrekturen
\hyphenation{Auf-triebs-pro-ble-me}
%\hyphenation{}
%\hyphenation{}


% Umbenennungen (babel) (siehe LaTeX-Begleiter, Abschn. 9.2.3)
%\addto\extrasgerman{\renewcommand{\figurename}{Abb.}}


% Listen
\newcommand{\bl}{\begin{list}{\textbullet}%         %% kleine Aufzaehlung/Liste
{\topsep0pt\partopsep0pt\itemsep0pt\parsep0pt\leftmargin1.5em\labelwidth1em\labelsep0.5em}}
\newcommand{\el}{\end{list}}
\newcommand{\cit}[1]{\textit{\cite{#1}}}

%Makros f�r eine erleichterte Eingabe wiederkehrender Befehle:
\newcommand{\abb}[1]{Abbildung~\ref{#1}}   
\newcommand{\tab}[1]{Tabelle~\ref{#1}} 
\newcommand{\glg}[1]{Gleichung~\ref{#1}}
\newcommand{\kap}[1]{Kapitel~\ref{#1}}
\newcommand{\abschn}[1]{Abschnitt~\ref{#1}}
\newcommand{\anh}[1]{Anhang~\ref{#1}}



\begin{document}
% Titelseite
\begin{titlepage}
 \centering

\begin{table}[htbp]
 \begin{center}
 \vspace{-0.5cm}
% \renewcommand{\arraystretch}{1.5}
  \begin{tabular}{lcr} 
    \parbox{0.45\textwidth}{\mbox{ }} & \parbox{0.13\textwidth}{\mbox{ }} & \parbox{0.45\textwidth}{\mbox{ }} \\
    \hspace*{-2.0cm}
    \includegraphics[width=0.42\textwidth]{./figures/TUBraunschweig_4C.pdf} & 
     &  \\ %empty
  \end{tabular}
 \end{center}
\end{table}


 \vspace*{2.0cm}

 \textbf{\large Projektarbeit}


 \vspace*{1.5cm}
 
 \textbf{\LARGE D-shaped bluff body experiment} \\[0.5ex]
 


 \vspace*{1.5cm}

 \textbf{\large cand. mach. Vorname Familienname} \\[0.5ex]
 \textbf{\large Matrikelnummer 1234567}


 \vspace*{7.0cm}

 \begin{table}[htbp]
  \begin{center}
%  \renewcommand{\arraystretch}{1.5}
   \begin{tabular}{rl} 
     \parbox{0.33\textwidth}{\mbox{ }} & \parbox{0.66\textwidth}{\mbox{ }} \\
     Ausgegeben: & Jun.-Prof. Dr.-Ing. D. Ko{\v z}ulovi{\' c} \\
                 & Institut f\" ur Str\" omungsmechanik \\
                 & Institutsleiter: Prof. Dr.-Ing. R. Radespiel \\
                 & Technische Universit\" at Braunschweig \\
                 &  \\
       Betreuer: & Dipl.-Ing. X Y, (externe Firma) \\
                 & Dipl.-Ing. X Y, (TU Braunschweig) \\
                 &  \\
 (Erstellt bei:) & (Externe Firma, Stadt) \\
                 &  \\
 Ver"offentlichung: & Monat Jahr \\
   \end{tabular}
  \end{center}
 \end{table}



\end{titlepage}


\include{00_empty_page}

\pagenumbering{roman}\setcounter{page}{1} 


%% Eidesstattliche Erklaerung (fuer DA, MA und BA)
%\include{eid_erkl}
%\include{empty_page}


% Uebersicht
\chapter*{Zusammenfassung}\label{s:uebersicht}

In dieser Arbeit wird die m"ogliche Widerstandsreduktion an einem D-f"ormigen Stumpfk"orper durch eine periodische hybride Aktuationsmethode untersucht.
Die aktive Str"omungsbeeinflussung kombiniert dabei die Grenzschichtbeeinflussung durch sich drehende, gezahnte Zylinder mit der Ausblasung von Coand\^{a}-Jets an der Hinterkante des Modells.



\include{00_empty_page}


% Inhaltsverzeichnis
\fancyhead[RE]{Inhaltsverzeichnis}
\fancyhead[LO]{Inhaltsverzeichnis}
{ 
   \renewcommand{\baselinestretch}{0.85}    %% Zeilenabstand / TOC
   \small\normalsize                        %% neuen Zeilenabstand aktivieren
   \tableofcontents
}
% ATTENTION: comment out if contents sides are even number
%\include{00_empty_page}


% Nomenklatur
\newpage
\fancyhead[RE]{Nomenklatur}
\fancyhead[LO]{Nomenklatur}
\chapter*{Nomenklatur}
\addcontentsline{toc}{chapter}{Nomenklatur}


\subsection*{Lateinische Bezeichnungen}
\begin{tabbing}
\hspace*{2cm}\=\kill
%$AVDR$ \> axiales Stromdichteverh\"altnis \\[0.2ex]
$Str$ \> Strouhal-Zahl \\[0.2ex]
\end{tabbing}



\subsection*{Griechische Bezeichnungen}
\begin{tabbing}
\hspace*{2cm}\=\kill
%$\beta$ \> Winkel in Umfangsrichtung \\[0.2ex]
\end{tabbing}



\subsection*{Indizes}
\begin{tabbing}
\hspace*{2cm}\=\kill
%$ax$ \> in axiale Richtung \\[0.2ex]
\end{tabbing}



\subsection*{Abk\"urzungen}
\begin{tabbing}
\hspace*{2cm}\=\kill
%$CFD$ \> \underline{C}omputational \underline{F}luid \underline{D}ynamics \\[0.2ex]
$NB$ \> \underline{N}ora M. \underline{B}ierwagen \\[0.2ex]
$TG$ \> \underline{T}im \underline{G}otzel \\[0.2ex]
$AK$ \> \underline{A}miriman \underline{K}ianfar \\[0.2ex]
$KK$ \> \underline{K}ebria \underline{K}iani \\[0.2ex]
$FT$ \> \underline{F}lorian \underline{T}imm \\[0.2ex]
$MATLAB$ \> kommerzielle Software zur L"osung mathematischer Probleme\\[0.2ex]
$rpm$ \> Umdrehungen pro Minute \\[0.2ex]
\end{tabbing}




% alle Kapitel
\pagenumbering{arabic}\setcounter{page}{1}
\fancyhead[RE]{\nouppercase{\leftmark}}
\fancyhead[LO]{\rightmark}
\chapter{Einleitung}\label{s:einleitung}

kurze Einleitung Stumpfk"orper\\
aufbauend auf Masterarbei\\
Ziel der Arbeit\\

Da es sich bei diesem Dokument um eine Projektarbeit handelt, an der insgesamt f"unf Personen mitgewirkt haben, stehen hinter jeder Kapitel- bzw. Unterkapitel"uberschrift die Initialen des Autors. In \tab{tab:initialien} ist eine Aufschl"usselung der Initialien gegeben.

\begin{table}[h]
	\centering
	\begin{tabular}{lr}
		\toprule
		Name & Initialien\\
		\midrule
		Nora M. Bierwagen & NB\\
		Tim Gotzel & TG\\
		Amiriman Kianfar & AK\\
		Kebria Kiani & KK\\
		Florian Timm & FT\\
		\bottomrule
	\end{tabular}
	\caption{Initialien}
	\label{tab:initialien}
\end{table}

\chapter{Grundlagen}\label{s:grundlagen}

\section{Stumpfk\"orperaerodynamik (TG)}

%	Stumpfkörper
%	D-Stumpfkörper
%	Strömungsfeld
%	Ablösung
%	Totwasser
%	Wirbelschichten
%	Nachlauf
%	Praktische Bedeutung
%	Ggf. Vergleich schlanke Körper

Im folgenden Kapitel wird der Begriff der Stumpfk"orper in Abgrenzung zu den schlanken K"orpern eingef"uhrt. Dabei soll im Besonderen auf das charakteristische Str"omungsfeld eingegangen und die Charakteristiken des Nachlaufs spezifiziert werden.

\subsection{Geometrische Einordnung}
Ein stumpfer K"orper in einer Anstr"omung differenziert sich geometrisch von einem schlanken insofern, dass er eine signifikante Dicke quer zur Anstr"omung aufweist, welche in vergleichbarer Gr"o\ss{}enordnung wie die Abmessungen parallel zur Anstr"omung liegt. Als Ma\ss{} kann das Dickenverh"altnis $\sigma$ als Kehrwert des Schlankheitsgrades $\lambda$ herangezogen werden, welches das Verh"altnis von Dicke zu Breite wiedergibt:

\begin{align}
\sigma = \frac{1}{\lambda} = \frac{d}{l}
\end{align}

Wie in \abb{fig:HuchoDV} zu sehen ist, ver"andert sich das Str"omungsbild ma\ss{}geblich mit dem Dickenverh"altnis $\sigma$, wobei der "Ubergang von schlanken zu stumpfen K"orpern flie\ss{}end ist.

\begin{figure}[h]
	\centering
	\includegraphics[width=0.5\textwidth]{HuchoDickenverhaltnis.jpg}
	\caption{elliptische Zylinder unterschiedlicher Dickenverh"altnisse im Rauchkanal \cite{Hucho.2011} \\ oben $\sigma = 0,13$ \\ Mitte $\sigma = 0,26$ \\ unten $\sigma = 0,5$ }
	\label{fig:HuchoDV}
\end{figure}

Neben dem Dickenver"altnis kann ein diskontinuierlicher Verlauf der K"opergeometrie, wie dies beispielsweise bei  der ausgepr"agten Hinterkanten eines Fahrzeughecks der Fall ist, als geometrische Charakterisierung eines stumpfen K"orpers herangezogen werden. 

\subsection{Str"omungsbild}
\label{sec:Stromungsbild}
Bei der Umstr"omung des K"orpers kommt es zur Ausbildung einer Grenzschicht. Ein Teil der kinetischen Energie der Grenzschichtstr"omung wird durch Reibung an der Wand dissipiert. Die Geometrie eines stumpfen K"orpers, wie in \abb{fig:HuchoDV} dargestellt, f"uhrt gem"a\ss{} Bernoulli zu einer Absenkung des statischen Drucks bis zur dicksten Stelle. Hinter dieser steigt der statische Druck wieder an, wobei die durch Reibung verringerte kinetische Energie nicht mehr ausreicht, um gegen diesen anzustr"omen. Ist die kinetische Energie vollends in Druck umgewandelt, kommt es zur R"uckstr"omung, wobei die Grenzschicht abl"ost \cite{Hucho.2011}.\\
Der gleichen Argumentation folgend ist der Abl"osepunkt an einer gegebenenfalls vorhandenen diskontinuierlichen Stelle der K"orpergeometrie lokalisiert. In der Praxis ist dies h"aufig eine Hinterkante. 

Das Abl"osen hat die Ausbildung eines Totwassers zur Folge, in dessen Gebiet sich das Fluid bedingt durch Z"ahigkeitseffekte verwirbelt und Wirbelschichten ausbildet. Wie man in \abb{fig:HuchoStumpf} sehen kann, ist das Totwasser ein Charakteristikum des Str"omungsbildes stumpfer K"orper. Im Vergleich dazu ist dieses Gebiet beim Str"omungsbild schlanker K"orper, wie in \abb{fig:HuchoSchlank} zu sehen, nicht vorhanden, da hier ein nahezu st"orungsfreies Abstr"omen m"oglich ist.

\begin{figure}[h]
	\centering
	\includegraphics[width=0.5\textwidth]{HuchoschlankerKorper.jpg}
	\caption{Stromlinienbild eines schlanken K"orpers im Rauchkanal \cite{Hucho.2011}}
	\label{fig:HuchoSchlank}
\end{figure}

\begin{figure}[h]
	\centering
	\includegraphics[width=0.5\textwidth]{HuchoStumpferKorper.jpg}
	\caption{Stromlinienbild eines stumpfen K"orpers im Rauchkanal \cite{Hucho.2011}}
	\label{fig:HuchoStumpf}
\end{figure}

Innerhalb des Totwassers existiert eine instation"are periodische Str"omung, welche durch Druckschwankungen zu oszillierende Abl"osungen f"uhrt. Diese Oszillation wei\ss{}t eine charakteristische Frequenz auf und wird durch die Stouhal-Zahl ${Sr}$ beschrieben. 

Die Wirbel innerhalb der turbulenten Str"omung  zerfallen kaskadenartig in kleinere Wirbel und dissipieren dabei ihre Energie in W"arme, bis sie sich g"anzlich aufl"osen und sich erneut ein laminares Str"omungsprofil ausbildet. Dennoch ergibt sich im Nachlauf des stumpfen K"orpers eine Delle im Geschwindigkeitsfeld. 

\begin{figure}[h]
	\centering
	\includegraphics[width=0.7\textwidth]{Nachlauf.jpg}
	\caption{Nachlauf eines a) schlanken K"orpers und eines b) stumpfen K"orpers \cite{Hucho.2011}}
	\label{fig:Nachlauf}
\end{figure}

\subsection{Praktische Bedeutung der Stumpfk"orper}
tbd


\section{Coand\^{a}-Effekt (TG)}
%	Umströmung gekrümmter Flächen
%	Haftbedingung
%	Typische Strömungsbilder
%	technische Anwendung 

Der Coand\^{a}-Effekt tritt auf, wenn ein Strahl entlang einer K"orperkontur str"omt. Anders als die bisher betrachtete Str"omung, kann die sogenannte Coand\^{a}-Str"omung des Strahles der Kontur einer konvexen Rundung folgen ohne abzul"osen. Bei der normalen Anstr"omung f"uhrt die konvexe Rundung nach Bernoulli zu einer Verlangsamung der Str"omung und demgem"a\ss{} einer Druckerh"ohung, was eine R"uckstr"omung und Abl"osung zur Folge hat. Dies wurde bereits in \ref{sec:Stromungsbild} diskutiert.

Da es sich bei der Coand\^{a}-Str"omung um einen Strahl handelt, gibt es zur Grenzschicht eine zus"atzliche Reibungsschicht zum umgebenden Medium. Dies wird in \abb{fig:coanda} gezeigt. Da das Umgebungsmedium ruht gibt es gem"a\ss{} Bernoulli keinen Druckanstieg entlang der konvexen Rundung, die Grenzschicht bleibt stabil. Aus diesem Grund haftet die Coand\^{a}-Str"omung l"anger an K"orperkontur an.

\begin{figure}[h]
	\centering
	\includegraphics[width=0.5\textwidth]{coanda.jpg}
	\caption{Skizze zum Coand\^{a}-Effekt \cite{Stadlberger.2016}}
	\label{fig:coanda}
\end{figure}





%%%%%%%%%%%%%%%%%%%%%%%%%%%%%%%%NORAS REICH ;) %%%%%%%%%%%%%%%%%%%%%%%%%%%%%%%%%%%%%%%%%%%%%%%%%
\section{Aktive Str\"omungsbeeinflussung (NB)}

Stumpfe K"orper haben meist ein abruptes Ende, an dem sich str"omungsmechanische Nachteile ergeben. Diese versucht man durch Anpassung der Geometrie des K"orpers oder durch die strukturelle Ver"anderung des Todwassers auszugleichen. Ziel ist es den Basisdruck anzuheben und dar"uber den Druckwiderstand des K"orpers zu verringern \cite{Hucho.2011}.\\
%hier evtl ein paar passive verfahren
Im Rahmen dieser Arbeit wird sich auf eine aktive Str"omungsbeeinflussung konzentriert, weshalb im folgenden einige bis jetzt realisiete Verfahren vorgestellt werden.\\

%-------------------------------------------------------------------------------------
Bearman \cite{Hucho.2011} hat als einer der ersten die aktive Str"omungsbeeinflussung nachgewiesen. \abb{fig:Bearman} zeigt das verwendete Stumpfk"orpermodell. Dabei ist als Besonderheit auf die por"ose Basis hinzuweisen, durch die zus"atzlich Luft am Ende des K"orpers ausgesto\ss{}en wurde.
\begin{figure}[h]
	\centering
	\includegraphics[width=0.5\textwidth]{KorperBearman.jpg}
	\caption{Stumpfk"orper mit Ausblasung von Bearman \cite{Hucho.2011}}
	\label{fig:Bearman}
\end{figure}\\
Die austretende Luft sorgte daf"ur, dass die Str"omungsabl"osung vom K"orperende weggeschoben wurde. Durch die erst sp"ater stattfindende Verwirbelung, f"allt der Widerstand des K"orpers ab.

%-------------------------------------------------------------------------------------
Geropp \cite{Geropp.2000} hat Experimente zur Einblasung am Ende eines Kraftfahrzeuges "uber zwei Schlitze mit Nutzung des Coand\^{a}-Effekts gemacht (\abb{fig:Geropp}).
\begin{figure}[h]
	\centering
	\includegraphics[width=0.5\textwidth]{KorperGeropp.jpg}
	\caption{Stumpfk"orper mit Ausblasung von Geropp \cite{Geropp.2000}}
	\label{fig:Geropp}
\end{figure}\\
Ergebis ist, dass die Ausblasung bei hohen Geschwindigkeiten erfolgen muss, um die Genzschicht zu beeinflussen. Durch den Coand\^{a}-Effekt wird die eingeblasene Luft in das Todwasser umgelenkt, wo sie wieder abgesagt wird. Dadurch wird der Druck hinter dem Fahrzeug erh"oht und der Gesamtwiderstand verrringert. Die Experimente haben gezeigt, dass eine Druckerh"ohung von 50\% und eine Widerstandsverringerung um 10\% m"oglich ist. Au\ss{}erdem wurde ein Energievorteil f"ur moderate Ausblasgeschwindigkeiten mathematisch bestimmt.\\

%--------------------------------------------------------------------------------------
In \cite{Barros.2016} wurde zus"atzlich zu den vorherig beschriebenen Verfahren die Ausblassung gepulst durchgef"uhrt. Dabei soll der Einfluss von Frequenz und Amplitude auf das Widerstandsverhalten untersucht werden.
\begin{figure}[h]
	\centering
	\includegraphics[width=0.8\textwidth]{KorperBarros.jpg}
	\caption{Ausblasung von Barros \cite{Barros.2016}}
	\label{fig:Barros}
\end{figure}\\
In \abb{fig:Barros} ist der schematische Aufbau der gepulsten Ausblasung dargestellt. Diese wird "uber Ventile realisiert, die eine Rechteckkurve mit einem duty cycle (weiter Erl"auterung in \kap{rotierendeWalze}) von 40\% erzeugen. Direkt unter der Ausblasestelle wird zus"atzlich noch eine Coand\^{a}-Fl"ache befestigt.\\
Mit steigenden Frequenzen, aber auch mit einer steigenden Amplitude, wurde eine Umlenkung der Grenzschicht beobachtet. "Uber eine gepulste Einblasung, nahe der nat"urlichen Abl"osefrequenz der Str"omung, konnte der Widerstand am meisten (10\%) gesenkt werden. Bei zus"atzlicher Nutzung der Coand\^{a}-Fl"ache kommt man sogar auf 20\% Widerstandsreduzierung.

%-------------------------------------------------------------------------------------
Modi \cite{MODI.1991} versucht durch drehende Zylinder an einem Truck den Widerstand zu reduzieren.
\begin{figure}[h]
	\centering
	\includegraphics[width=0.6\textwidth]{KorperModi.jpg}
	\caption{Truckmodell von Modi \cite{MODI.1991}}
	\label{fig:Modi}
\end{figure}\\
Dazu sind bei den ersten Windkanalversuchen die Zylinder angeordnet, wie in \abb{fig:Modi} dargestellt. Bei dem ersten Versuch wird die Rauhigkeiten der Zylinder mit durchvariiert. Es gibt einen glatten Zylinder, einen mit einer Rauhigkeit von 40 und einen mit 80. Au\ss{}erdem wird das Verh"altnis der Geschwindigkeiten der Zylinder \(U_c\) bzgl. der Anstr"omgeschwindigkeit \(U\) f"ur alle drei F"alle variiert. Daraus ergeben sich die Widerstandsreduktionen in \tab{tab:Modi}.
\begin{table}[h!]
	\centering
	\begin{tabular}{lrr}
		\toprule
		Zylinder & Widerstandsreduktion [\%] & \(U_c/U\)\\
		\midrule
		glatt & 5 & 2\\
		Rauhigkeit 80 & 10 & 2.1\\
		Rauhigkeit 40 & 13 & 2.1\\
		\bottomrule
	\end{tabular}
	\caption{Widerstandsreduktion bei Modi}
	\label{tab:Modi}
\end{table}\\
Da der hintere Zylinder keinen Impuls in die Grenzschicht einbringen kann und die Versuche das auch best"atigt haben, wurde ein zweites Experiment mit anderer Konfiguration durchgef"uhrt. Dabei wurde ein Zylinder mit spiralf"ormiger Rille in der Oberfl"ache und einer mit einer Vielkeil-Verzahnung, deren Rillen parallel zur Drehachse verlaufen, verwendet. Der erste Zylinder sitzt wie in \abb{fig:Modi} dargestellt, der zweite wurde ans Ende des ersten Drittels der Truckoberseite positioniert.\\
Der spiralf"ormige Zylinder erzielte das gleiche Ergebnis, wie der Zylinder mit einer Rauhigkeit von 40 im ersten Experiment. Der Vielkeil-Verzahnungs Zylinder hat allerdings einen gro\ss{}en Einfluss auf den Widerstand. Wenn man nur den vorderen Zylinder betrachtet k"onnen 29\% Reduktion erreicht werden, beide erreichen bis zu 41\%.

%------------------------------------------------------------------------------------
Alle bisher vorgestellten Verfahren haben nur eine Steuerung des Vorgangs betrachtet. In \cite{Henning.2008} wird jetzt zus"atzlich eine Regelung des Mechanismuses der Str"omungsbeeinflussung beachtet. Dabei m"ochte man "au\ss{}ere St"orungen mit ber"ucksichtigen, die zum Beispiel sich gegenseitig beeinflussende Kraftfahrzeuge aufeinander haben.\\
Im Rahmen der Arbeit wurden unterschiedliche K"orper (\abb{fig:Henning}) analysiert.
\begin{figure}[h]
	\centering
	\includegraphics[width=0.8\textwidth]{KorperHenning.jpg}
	\caption{betrachtete Modelle f"ur Auslegung der Regelung \cite{Henning.2008}}
	\label{fig:Henning}
\end{figure}\\
An der r"uck"arts gewandten Stufe wurde erfolgreich die Wideranlegel"ange "uber einen segmentierten Schlitz an der Stufenkante geregelt. Au\ss{}erdem konnte eine Unterdr"uckung von St"orungen erreicht werden. Das Ganze wurde "uber eine Robuste Regelung realisiert.\\
Am stumpfen K"orper wurde mit Hilfe einer Phasenregelung an Ober- und Unterseite eine Widerstandsreduzierung von bis zu 15\% erreicht.\\
Die Tandemkonfiguration wurde im Rahmen einer Machbarkeitsstudie untersucht und f"ur zuk"unftige Arbeiten als sinnvoll betrachtet. Dabei geht es um die St"oreinfl"usse, die der erste K"orper auf den zweiten hat und wie dieser die St"orung "uber eine Regelung beseitigen kann, sodass auch beim zweiten K"orper eine Widerstandsreduzierung m"oglich ist.

\chapter{Konstruktion der rotierenden Walzen (NB)}
\label{s:rotierendeWalzen}
Bevor das Experiment im Windkanal stattfinden kann, m"ussen die rotierenden Walzen f"ur den Stumpfk"orper konstuiert werden. Die Walzen sitzen am Ende des K"orpers, wie in \abb{fig:modelschema} gezeigt. Es sind drei unterschiedliche Walzenpaare entstanden: ein glattes Walzenpaar, ein gezahntes Walzenpaar mit einem duty cycle von 33\% und ein gezahntes Walzenpaar mit einem duty cycle von 50 \%.\\
%Die rotierenden Walzen sind jeweils auf einer Welle gelagert und werden von Elektromotoren angetrieben. Die Elektromotoren haben eine maximale Drehrate von 3650 Umdrehungenpro Minute und eine minimale Drehrate von um die 100 Umdrehungen pro Minute.

\begin{figure}[h]
	\centering
	\includegraphics[width=0.8\textwidth]{ModellSchema.jpg}
	\caption{schematische Darstellung des Versuchsmodells}
	\label{fig:modelschema}
\end{figure}
Die rotierenden Walzen erf"ullen die Aufgabe, der gepulsten Einblasung in die Str"omung am Ende des Stumpfk"orpers.\\
Die Walzen bestehen aus einer Aluminium Innenewelle und einem mit Presspassung verbundenen Teflonrohr. In dem Teflonrohr ist die entscheidende Zahngeometire eingebracht. F"ur Konstruktion des Teflonrohrs mussten folgende Aspekte betrachtet werden:
\begin{enumerate}
	\item Zahnform 
	\item Anzahl der Z"ahne
	\item Zahn"offnung 
\end{enumerate}

%----------------------------------------------------------------------------------
\section{Zahnform}
F"ur die Wahl einer Zahnform muss erst das auf die Str"omung aufgebrachte Signal festgelegt werden. Als Signale kommen daf"ur unterschiedliche Funktionene in Frage: Sinus-Funktion, Dirac-Impuls, Heaviside-Funktion usw..\\
%Die Zahnform hat Auswirkungen auf das gepulste Signal, das in die Str"omung eingef"uhrt wird. Als Grundlagen f"ur die Signalform wurden folgende mathematische Funktionen (\abb{fig:function}) betrachtet.\\
%\begin{figure}[h]
%	\centering
%	\begin{subfigure}[c]{0.5\textwidth}		
%		\includegraphics[width=1\textwidth]{Sinus.jpg}
%	\begin{subfigure}[c]{0.5\textwidth}
%		\includegraphics[width=1\textwidth]{DeltaDistribution.jpg}
%	\end{subfigure}
%	\begin{subfigure}[c]{0.5\textwidth}
%		\includegraphics[width=1\textwidth]{Heaviside.jpg}
%	\end{subfigure}
%	\caption{mathematische Funktionen f"ur Zahnform}
%	\label{fig:function}
%\end{figure}\\
%F"ur die Sinus Funktion eignet sich eher eine andere Form der Str"omungsanregung, als die der drehenden Walzen. Diese kann besser "uber einen Null-Netto-Massenstrom-Jet-Aktuator \cite{Utturkar.2003} oder einen Lautsprecher dargestellt werden. Beide funktionieren "uber eine schwingende Membran, welche die Str"omung anregt.\\
%Ein Dirac-Impuls ist eine kurze Anregung der Str"omung. Die Zeit in der die Luft in die Str"omung eingeblasen wird, ist kurz im Vergleich zu der Zeit in der keine Anregung stattfindet. Damit k"onnte es zu einer nicht ausreichend gr\ss{}en Anregung kommen, die den gew"unschten Effekt nicht induziert.\\
%Eine Heaviside-Funktion stellt ein eindeutiges Signal dar, das entweder vollst"andig geschlossen oder ge"offnet ist. Somit ist eine klare Definition des Zustands m"oglich.\\
Bei der endg"ultigen Wahl eines Signals ist der fertigungstechnische Aspekt ein weiterer wichtiger Parameter, der in diesem Fall die Wahl des Signals entschieden hat. Als finales Wellendesign wurden die zwei Wellen aus \abb{fig:finalesdesign} gefertigt. Diese wurden gew"ahlt, da eine Fr"asbearbeitung des Teflonrohrs zu str"omungsmechanisch ung"unstigen Effekten gef"uhrt h"atte. Der Fr"aser hat immer eine endliche Breite, sodass die Str"omung durch die eventuell auftretenden minimalen Kanten zwischen den einzelnen Fr"asbahnen gest"ort werden k"onnte. Somit wurde sich f"ur eine Fertigung auf der Drehmaschine entschieden. Dabei wurden die Zahnt"aler "uber eine exzentrische Einspannung erreicht. Die Wahl f"ur zwei Wellen wird im Folgenden n"aher betrachtet.
\begin{figure}[h]
	\centering
	\begin{subfigure}[c]{0.4\textwidth}		
		\includegraphics[width=0.8\textwidth]{Walze1Graphik.jpg}
	\end{subfigure}
	\begin{subfigure}[c]{0.4\textwidth}
		\includegraphics[width=0.8\textwidth]{Walze2Graphik.jpg}
	\end{subfigure}
	\caption{Querschnitt durch die finalen Walzen}
	\label{fig:finalesdesign}
\end{figure}\\

Die Walzen Formen lassen sich "uber mehrere Kreisfunktion auf unterschiedlichen Intervallen darstellen (verschiedene Farben in \abb{fig:finalesdesign}). Die linke Walze kann beschrieben werden "uber \glg{eq:Walze1}.
\begin{align}
	{f_1(x)}=\pm\sqrt{9,15^{2}-(x-9,45)^{2}}\,\,\,\,&x\in[0,3; 2,42] \label{eq:Walze1}\\
	{g_1(x)}=\pm\sqrt{8,3^{2}-(x-8,3)^{2}}\,\,\,\,&x\in[2,42; 14,18] \nonumber\\
	{h_1(x)}=\pm\sqrt{9,15^{2}-(x-7,15)^{2}}\,\,\,\,&x\in[14,18; 16] \nonumber
\end{align}\\
Die rechte Walze kann beschrieben werden "uber \glg{eq:Walze2}.
\begin{align}
	{f_2(x)}=\pm\sqrt{8,48^{2}-(x-8,78)^{2}}\,\,\,\,&x\in[0,3; 5,39] \label{eq:Walze2}\\
	{g_2(x)}=\pm\sqrt{8,3^{2}-(x-8,3)^{2}}\,\,\,\,&x\in[5,39; 10,61] \nonumber\\
	{h_2(x)}=\pm\sqrt{8,48^{2}-(x-7,82)^{2}}\,\,\,\,&x\in[10,61; 16] \nonumber
\end{align}\\
Aus der Form der Walzen, die die Zahnform darstellen, l"asst sich r"uckwirkend auf die Signalform schlie\ss{}en. Das Signal ist in \abb{fig:spaltverlauf} dargestellt. Ein Wert von \SI{0,3}{\milli\meter} entspricht dabei einem offenen Signal, d.h. es wird Luft in den Spalt eingeblasen. Bei einem Wert von \SI{0}{\milli\meter} findet keine Einblasung statt. In den Graphiken ist ein Signalverlauf f"ur eine Viertelumdrehung der Walze dargestellt. Aufgrund von Symmetrie folgt im weiteren ein spiegelverkehrter Verlauf und danach eine periodische Fortsetzung, wie in \abb{fig:signal} dargestellt.
\begin{figure}[h]
	\centering
	\begin{subfigure}[c]{0.5\textwidth}		
		\includegraphics[width=1\textwidth]{Spaltverlauf1.jpg}
	\end{subfigure}
	\begin{subfigure}[c]{0.5\textwidth}
		\includegraphics[width=1\textwidth]{Spaltverlauf2.jpg}
	\end{subfigure}
	\caption{Signalverlauf der Walzen}
	\label{fig:spaltverlauf}
\end{figure}\\
Der linerare Verlauf der ersten Waltze l"asst sich ann"ahern "uber \glg{eq:Spalthoehe1} und der der zweiten Walze "uber \glg{eq:Spalthoehe2}.
\begin{align}
{l_1(x)}=-0.14x+0.3388
\label{eq:Spalthoehe1}
\end{align}
\begin{align}
{l_2(x)}=-0.059x+0.31801
\label{eq:Spalthoehe2}
\end{align}\\
\begin{figure}[h]
	\centering
	\begin{subfigure}[c]{0.5\textwidth}		
		\includegraphics[width=1.1\textwidth]{SignalWalze1.jpg}
	\end{subfigure}
	\begin{subfigure}[c]{0.5\textwidth}
		\includegraphics[width=1.1\textwidth]{SignalWalze2.jpg}
	\end{subfigure}
	\caption{Periodische Signale der Walzen}
	\label{fig:signal}
\end{figure}\\


%--------------------------------------------------------------------------------------
\section{Anzahl der Z"ahne}
Als zweites wird im folgenden auf die Entscheidung der Anzahl der Z"ahne detailierter eingegangen. Ein ausschlaggebender Punkt ist dabei, dass die Einblasung mit einer Frequenz in Umgebung der Abl"osefrequenz der Str"omung durchgef"uhrt wird, besonders interessant ist eine Frequenz leicht oberhalb der Abl"osefrequenz (siehe \cite{Oswald.2017}). Au\ss{}erdem soll die Drehzahl der Elektromotoren nicht "uber- bzw. unterschritten werden. Diese liegt maximal bei 3650 Umdrehungen pro Minute und minimal bei 100 Umdrehungen pro Minute.\\
Die Abl"osefrequenz der Str"omung kann "uber die dimensionslose Strouhal-Zahl berechnet werden. Die Stouhal-Zahl ist nach \cite{Leder.1992} definiert als
\begin{align}
	{Str}=\frac{f*D}{U_{\infty}}	\label{eq:Str}
\end{align}
Hierbei ist f die Abl"osefrequenz der Str"omung, D die Profildicke im breitesten Querschnitt und $U_{\infty}$ die Anstr"omgeschwindigkeit des Windkanals.\\
Wenn man \glg{eq:Str} nach der gew"unschten Variable umstellt (siehe \glg{eq:nachfumgestellt}) und die gegebenen Werte aus \tab{tab:Modellwerte} einsetzt, erh"alt man
\begin{align}
	{f}=\frac{Str*U_{\infty}}{D}
		=\frac{0,23*\SI{15}{\meter\per\second}}{\SI{0,0534}{\meter}}
		=\SI{64,61}{\hertz}
		\label{eq:nachfumgestellt}
\end{align}
\begin{table}[h!]
	\centering
	\begin{tabular}{lr}
		\toprule
		Parameter & Wert\\
		\midrule
		Strouhal-Zahl & 0,23\\
		Anstr"omgeschwindigkeit & \SI{15}{\meter\per\second}\\
		Profildicke & \SI{53,4}{\milli\meter}\\
		\bottomrule
	\end{tabular}
	\caption{Modellwerte f"ur Frequenzberechnung}
	\label{tab:Modellwerte}
\end{table}\\
Aus der Frequenz, die mindestens erreicht werden soll, kann nun berechnet werden, wie schnell sich die Walze mit welcher Z"ahnezahl drehen muss. Das Ergebnis f"ur unterschiedliche Z"ahne findet sich in \tab{tab:zahnezahl}.\\
\begin{table}[h]
	\centering
	\begin{tabular}{lrr}
		\toprule
		Anzahl der Z"ahne & Drehgeschwindigkeit der Welle [Hz] & Drehgeschwindigkeit der Welle [rpm]\\
		\midrule
		1 & 64,61 & 3876,6\\
		2 & 32,30 & 1938,0\\
		4 & 16,15 & 969,0\\
		6 & 10,77 & 646,2\\
		8 & 8,08 & 484,8\\
		10 & 6,46 & 387,6\\
		\bottomrule
	\end{tabular}\\
	\caption{Z"ahnezahlen mit zugeh"origen Frequenzen}
	\label{tab:zahnezahl}
\end{table}

Aufgrund von Symmetrie und um dem m"oglichen entstehen einer Unwucht entgegen zu wirken, wurde sich f"ur eine gerade Z"ahnezahl entschieden. Durch die ben"otigte ausrechend gro\ss{}e L"ucke zwischen den Z"ahnen um einen effektiven Luftaussto\ss{} zu gew"ahrleisten, wurde sich f"ur eine Z"ahnezahl von zwei pro Walze entschieden.\\

%-----------------------------------------------------------------------------------
\section{Zahn"offnung}
Als dritten Aspekt der Zahngestaltung wird im folgendem die Zahn"offnung betrachtet. Es wurden zwei Walzen mit zwei unterschiedlichen duty cyclen gefertigt. Der duty cycle gibt den prozentualen Anteil der Zahn"offnung auf die gesamtm"ogliche Zahn"offnung (keine Z"ahne) an. Die Berechnungen des duty cycles f"ur die erste Walze ist "uber Integration von \glg{eq:Walze1} und f"ur die zweite Walze von \glg{eq:Walze2} an einer Viertelwalze erfolgt. F"ur die Berechnungen wird das MATLAB Skript aus \abb{fig:Integrationsskript} verwendet. Somit ergibt sich f"ur die erste Walze ein duty cycle von 33\% und f"ur die zweite von 50\%.
\begin{figure}[h]
	\centering
	\includegraphics[width=0.8\textwidth]{Integrationsskript.jpg}
	\caption{MATLAB Skript zur Berechnung der Zahn"offnung}
	\label{fig:Integrationsskript}
\end{figure}\\


\chapter{Widerstandsbestimmung}\label{s:widerstandsbestimmung}


Text.
\section{mathematisches Modell}

\section{Implementierung}


\chapter{Windkanalversuche}\label{s:versuche}

\section{Windkanal}

\section{Versuchsaufbau}

\section{Messeinrichtung}

\section{Versuchsdurchf\"uhrung}



\chapter{Versuchsauswertung}
\label{s:auswertung}
\section{Einleitung}
\label{s:einleitungAusw}
Um den Einfluss der widerstandsreduzierenden Effekte bestimmen zu k\"onnen, werden beide Wellenpaare bei 3  verschiedenen Konstellationen getestet. So kann die Effektivit\"at von der Ausblasung und der Rotation erst separat und dann in Kombination untersucht werden.
Alle Messungen werden bei einer vorbestimmten Reylondszahl von 50000 durchgef\"uhrt.
Im folgenden Diagramm ist die Nachlaufdelle bei dem D-f\"ormigen Stumpfk\"orper, ohne jegliche widerstandsreduzierenden Ma\ss{}nahmen zu sehen. 
Es wird erwartet, dass die Nachlaufdelle, durch die zu untersuchenden Effekte kleiner wird.
\begin{figure}[h]
	\centering
	\includegraphics[width=0.75\textwidth]{Druckverlaeufe_Nachlauf_ohneAkt.jpg}
	\caption{ $\Delta P$ \"uber y/d ohne Aktuierung }
	\label{fig:Deltap-y/d_ohne_Aktuation}
\end{figure}

\section{Konstellation 1: Reine Rotation (KK)}
\label{s:ReineRotation}

In  \abb{fig:Cw-n_Rein_Konf+2} sind die Widerstandsbeiwerte von beiden Wellenpaaren bei verschiedenen Wellendrehzahlen zu sehen.
Die nat\"urliche Abl\"osefrequenz wird bei einer Drehzahl von 1938 1/min  erreicht. F\"ur eine gute Vergleichbarkeit wird der Bereich vor und insbesondere nach dieser Drehzahl, sowohl bei den glatten als auch bei den gezahnten Wellen, genauer und hochaufl\"osender Untersucht.
Abgesehen davon, dass die Konfiguration mit den ovalen bzw. gezahnten Wellen, in allen Drehzahlbereichen einen geringeren $C_{w}$-Wert im Vergleich zur Basiskonfiguration aufweisen, ist das Verhalten der beiden Kurven jedoch unterschiedlich.
\begin{figure}[h]
	\centering
	\includegraphics[width=0.5\textwidth]{Cw_ueber_n_ohneAB_beideKonf.jpg}
	\caption{ $C_{w}$-n reine Rotation beide Wellenpaare }
	\label{fig:Cw-n_Rein_Konf+2}
\end{figure}

W\"ahrend der $C_{w}$-Wert bei der Basiskonfiguration direkt nach der Impulsbeaufschlagung der Grenzschicht durch Rotation der Wellen, zu sinken anf\"angt, steigt der $C_{w}$-Wert der Konfiguration mit gezahnten Wellen bis zu einer Drehzahl von 1000 1/min. Erst ab einer Drehzahl von 1500 1/min, ist eine deutliche Widerstandsreduzierung zu erkennen. 
Anders als die Basiskonfiguration, bei der der minimale $C_{w}$-Wert im Bereich der erwarteten Abl\"osefrequenz erreicht wird, geschieht dies bei der Konfiguration 50 nach der Abl\"osefrequenz und zwar bei n= 2300. 
Beide Verl\"aufe steigen wieder, nachdem der niedrigste $C_{w}$-Wert erreicht wurde. Dennoch ist diese
Steigung mit den ovalen Wellen deutlich gr\"o\ss{}er.
Um von der Widerstandsreduzierung profitieren zu k\"onnen, muss noch genau verglichen werden, wie viel Leistung die jeweilige Konfiguration in Anspruch nimmt. Diese wird mit der kinetischen Leistung der Hauptstr\"omung genormt und als Leistungskoeffizient bezeichnet.

In \abb{fig:Cw-Cw0-CpowerM_reine} ist das $\Delta$ $C_{w}$-$C_{Power}$-Diagramm zu sehen. Es ist deutlich erkennbar, dass die ovalen Wellen, wesentlich mehr Energie ben\"otigen, um die Reibung zwischen der Teflonoberfl\"ache und dem Ausblaseschlitz zu \"uberwinden. Dazu kommt, dass die ovalen Wellen bei niedrigen Drehzahlen erst eine Widerstandssteigerung verursachen. 

Bei der Basiskonfiguration sinkt der $\Delta$ $C_{w}$-Wert, bereits bei geringer Energiezufuhr.  Nachdem der niedrigste Widerstandsbeiwert bei einer $C_{Power}$= 0,2 erreicht wurde f\"angt die Kurve an, lokal zu schwanken. Dies kommt durch die hohe lokale Drehzahlaufl\"osung im Bereich der Abl\"osefrequenz zustande. Laut Diagramm w\"urde sich nicht lohnen, die glatten Wellen mit einer h\"oheren Drehzahl als 1900  1/min zu drehen, da bei weiterer Energiezufuhr, keine Widerstandsreduzierung zu erkennen ist.
\begin{figure}[h]
	\centering
	\includegraphics[width=0.5\textwidth]{DeltaCw_ueberC_PowerM_ohne_AB_beideKonf.jpg}
	\caption{ $C_{w}$/$C_{w0}$ \"uber $C_{Power,M}$ f\"ur reine Rotation  }
	\label{fig:Cw-Cw0-CpowerM_reine}
\end{figure}
Bei der Konfiguration mit den ovalen Wellen muss deutlich mehr Leistung aufgebracht werden, um eine Widerstandsreduzierung im Vergleich zu der Konfiguration ohne Aktuieren zu erhalten.  Erst ab einem $C_{Power}$ von ca. 0,65 wird ein $C_{w}$/$C_{w0}$ erreicht, der kleiner als 1 ist. 
Danach sinkt der $C_{w}$-Wert stark un erreicht den kleinsten $C_{w}$-Wert bei einem $C_{Power}$= 0,84. 
Der minimale $C_{w}$/$C_{w0}$ ist zwar kleiner als bei der Basiskonfiguration und somit die Widerstandsreduzierung gr\"o\ss{}er, jedoch muss deutlich mehr Leistung aufgebracht werden.
\begin{table}[h]
	\centering
	\begin{tabular}{lrr}
		\toprule
		 & kleinster  $C_{w}$/$C_{w0}$ & $C_{Power}$ beim kleinsten $C_{w}$/$C_{w0}$ \\
		\midrule
		Basiskonfiguration & 0.93 & 0.2\\
		Konfiguration 50 & 0.91 & 0.84\\
		\bottomrule
	\end{tabular}\\
	\caption{ minimale  $C_{w}$-Werte und die entsprechenden $C_{Power}$ }
	\label{tab:minimalCw-Cpower}
\end{table}
Der Tabelle 6.5 zu entnehmen, muss man die Leistung um 420\% steigern, um mit den ovalen Wellen einen um ca. 2\% geringeren $C_{w}$/$C_{w0}$ als die Basiskonfiguration zu bekommen.

%%%%%%%%%%%%%%%%%%%%%%%%%%%%%%%%%%%%%%%%%%%%%%%%%%%%%%%%%%%%%%%%%%%%%%%%%%%%%%%%%%%%%%%%%%%%%%%%%%%%%%%%%%%

\section{Konstellation 2: Reine Ausblasung (AK)}
\label{s:reineAusblasung}
In \abschn{s:reineAusblasung} wurde das Profilmodell mit den glatten Welle im Windkanal eingebaut.Es gab keine Rotation bei den Wellen und der Versuch wurde mit den glatten Wellen ohne Aktuation durchgef\"uhrt. Der Volumenstrom wurde so eingestellt, dass
die folgenden $C_{\mu}$  Werte erreicht werden. Es wurde $C_{w}$ mit diesen vier verschiedenen $C_{\mu}$ Werte gemessen.Die Ergbnisse sind wie folgt:
\begin{table}[H]
	\centering
	\begin{tabular}{lr}
		\toprule
		$C_{\mu}$ & $C_{w}$ \\
		\midrule
		0.00 & 0.876\\
		0.16 & 0.606\\
		0.32 & 0.477\\
		0.48 & 0.399\\
		0.69 & 0.331\\
		\bottomrule
	\end{tabular}
	\caption{$C_{w}$ und $C_{\mu}$ mit der glatten Welle ohne Aktuation }
	\label{tab:Cw-Cmu_Kon1}
\end{table}


Die \abb{fig:Cw-Cmu_Konf1} zeigt die Funktion von Widerstandsbeiwert \"uber Impulsbeiwert f\"ur die glatte welle ohne Atuation:
\begin{figure}[h]
	\centering
	\includegraphics[width=0.5\textwidth]{Cw_ueber_Cmu_ohne_n.jpg}
	\caption{$C_{w}$  \"uber $C_{\mu}$ mit den glatten Wellen ohne Aktuation }
	\label{fig:Cw-Cmu_Konf1}
\end{figure}

Wie man im Diagramm betrachtet, ist es ein deutlicher Abfall von $C_{w}$ mit steigendem $C_{\mu}$.
F\"ur den Versuch mit den ovalen Wellen wurde $C_{\mu}$ Werte f\"ur die Verschiedenen Luftdr\"ucke berechnet. Unten sind die Auswertungen f\"ur $C_{w}$:
 
\begin{table}[h]
	\centering
	\begin{tabular}{lrr}
		\toprule
		Luftdruck [kPa] & $C_{\mu}$ & $C_{w}$ \\
		\midrule
		0 & 0.00 & 0.805\\
		1 & 0.06 & 0.610\\
		2 & 0.27 & 0.594\\
		3 & 0.42 & 0.577\\
		4 & 0.58 & 0.455\\
		\bottomrule
	\end{tabular}\\
	\caption{Luftdruck, $C_{w}$  und $C_{\mu}$ f\"ur die glatten und ovalen Wellen ohne Aktuation}
	\label{tab:Cw-Cmu_Konf1+2}
\end{table}

Die \abb{fig:Cw-Cmu_Konf1+2} zeigt den Verlauf $C_{w}$ \"uber $C_{\mu}$ f\"ur die beiden Konfigurationen:
\begin{figure}[h]
	\centering
	\includegraphics[width=0.5\textwidth]{Cw_ueber_Cmu_ohne_n_beideKonf.jpg}
	\caption{$C_{w}$  \"uber $C_{\mu}$ mit den glatten Wellen ohne Aktuation }
	\label{fig:Cw-Cmu_Konf1+2}
\end{figure}

Wie man im Diagramm sieht, f\"allt auch $C_{w}$  mit steigendem $C_{\mu}$  mit den ovalen Wellen ab.
\\
Im Vergleich mit den glatten Wellen sind die Widerstandsbeiwerte bei Impulskoeffizient im Bereich von 0 bis 0.16  in den ovalen Wellen niedriger als den glatten Wellen. Aber ab dem Punkt $C_{\mu}$= 0.16 sinkt der Widerstandsbeiwert der glatten Wellen st\"arker ab. Aus diesem Grund sind die glatten Wellen ohne Aktuation bei den gr\"o\ss{}eren Impulskoeffiziente zu empfehlen.

Dann werden die verschiedenen Widerstandsbeiwerte in Abh\"ungigkeit vom Leistungskoeffizient gemessen. Die Tabelle 6.3 ermittelt diese Werte f\"ur die Basiskonfiguration:
\begin{table}[h]
	\centering
	\begin{tabular}{lr}
		\toprule
		$C_{w}$/$C_{w0}$ & $C_{Power,Jet}$ \\
		\midrule
		1 & 0\\
		0.691 & 0.010\\
		0.544 & 0.283\\
		0.455 & 0.520\\
		0.378 & 0.896\\
		\bottomrule
	\end{tabular}
	\caption{$C_{w}$/$C_{w0}$ \"uber $C_{Power,Jet}$ f\"ur Basiskonfiguration }
	\label{tab:Cw/Cw0-CpJet_Kon1}
\end{table}

Und die Auswertungen f\"ur die Konfiguration mit den ovalen Wellen sind in der Tabelle 6.4:

\begin{table}[h]
	\centering
	\begin{tabular}{lr}
		\toprule
		$C_{w}$/$C_{w0}$ & $C_{Power,Jet}$ \\
		\midrule
		1 & 0\\
		0.748 & 0.149\\
		0.718 & 0.773\\
		0.566 & 1.90\\
		\bottomrule
	\end{tabular}
	\caption{$C_{w}$/$C_{w0}$ \"uber $C_{Power,Jet}$ f\"ur die ovalen Wellen }
	\label{tab:Cw/Cw0-CpJet_Kon2}
\end{table}

Die \abb{fig:Cw/Cw0-CpJet_Konf1+2} zeigt den Verlauf $C_{w}$/$C_{w0}$ \"uber $C_{Power,Jet}$  f\"ur die beiden Konfigurationen:
\begin{figure}[h]
	\centering
	\includegraphics[width=0.5\textwidth]{CwC0_ueber_CPowerJ_ohne_n_beideKonf.jpg}
	\caption{$C_{w}$/$C_{w0}$  \"uber $C_{Power,Jet}$ f\"r die beiden Konfigurationen}
	\label{fig:Cw/Cw0-CpJet_Konf1+2}
\end{figure}

Wie aus dem Diagramm ersichtlich ist, wird mehr Leistung ben\"otigt, um einen niedriegen Wiederstansbeiwert $C_{w}$  zu erreichen. Mit den ovalen Wellen wird mehr Leistung als den glatten Wellen ben\"otigt, um die gleiche Widerstandsreduzierung zu erzielen. Deswegen weist die Konfiguration mit den glatten Wellen in diesem Fall bessere Ergebnisse auf.
\newpage

\section{Konstellation 3: Kombinierte Aktuation (KK)}
\label{s:kombinierteAkt}
Nun wird bei beiden Konfigurationen, sowohl durch die Walzen als auch durch die Ausblasung aktuiert.\\

Die in \kap{s:reineAusblasung} verwendeten $C_{\mu}$-Werte f"ur die Konfiguration mit ovalen Wellen beziehen sich auf eine Wellenstellung, bei der der Spalt maximal ge"offnet ist, die Spalth"ohe also ca. 0,3\,mm betr"agt. Diese Position kann als einzige ge"offnete Wellenposition definiert eingestellt werden und dient der besseren Vergleichbarkeit mit der Basiskonfiguration. Wenn die Rotation hinzugenommen wird, "andert sich das $C_{\mu}$ allerdings und muss nach \glg{eq:momentum-coeff-oscill-final} berechnet werden. Die mit dem urspr"unglichen Impulskoeffizienten korrespondierenden neuen $C_{\mu}$-Werte sind in \tab{tab:Cmu Korrektur} festgehalten.

\begin{table}[H]
	\centering
	\begin{tabular}{lr}
		\toprule
		$C_{\mu}$ (reine Ausblasung) & $C_{\mu}$ (kombinierte Aktuation)\\
		\midrule
		0 & 0\\
		0,064 & 0,097\\
		0,27 & 0,231\\
		0,416 & 0,29\\
		0,582 & 0,327\\
		\bottomrule
	\end{tabular}
	\caption{Korrespondierende $C_{\mu}$-Werte f"ur reine Ausblasung und kombinierter Aktuation bei Konfiguration 50}
	\label{tab:Cmu Korrektur}
\end{table}


F"ur die Betrachtungen und Abbildungen wird aus "Ubersichtlichkeitsgr"unden das urspr"ungliche $C_{\mu}$ verwendet. Dieser Zusammenhang erschwert die Interpretation allerdings und muss stets im Hinterkopf behalten werden.

Die Vergleichbarkeit mit der Basiskonfiguration ist hier zus"atzlich dadurch eingeschr"ankt, dass nur bis zu mittleren Plenumsdr"ucken von 4\,kPa gemessen wurde und die Impulskoeffizienten f"ur diese Konfiguration nicht die, der Basiskonfiguration erreichen.\\
H"ohere Dr"ucke k"onnen gleichwohl nicht mitbetrachtet werden, da bei einem d"unnen verbleibenden Spalt die Dr"ucke deutlich "uber 9\,kPa liege, was wiederum in Jetgeschwindigkeiten oberhalb $Ma =$ $\frac{1}{3}$ m"undet. Die Annahme der inkompressiblen Jetstr"omung kann dann nicht mehr getroffen werden.

In \abb{fig:Cw/n bei Cmu RK} ist $C_{W}$ "uber n bei verschiedenen $C_{\mu}$ f"ur die Basiskonfiguration dargestellt.

\begin{figure}[h]
	\centering
	\includegraphics[width=0.5\textwidth]{Cw_ueber_n_fuer_verschiedeneCmu_RK.jpg}
	\caption{$C_{w}$ "uber $n$ bei verschiedenen $C_{\mu}$ f"ur die Basiskonfiguration}
	\label{fig:Cw/n bei Cmu RK}
\end{figure}

Au\ss{}er bei $C_{\mu}= 0,16$ sind alle anderen Verl"aufe in Abh"angigkeit von der Drehzahl "ahnlich.
Sie sinken mit einer "ahnlichen Rate bis zum Bereich der nat"urlichen Abl"osefrequenz und schwanken leicht wachsend danach.\\Mit einem $C_{\mu}= 0,16$ wird der $C_{W}$-Verlauf bis auf $n= 500$, wo der Widerstand minimal reduziert wird, immer oberhalb des $C_{W}$ der reinen Ausblasung gehalten. In diesem Fall bedeutet eine zus"atzliche Rotation mehr Energie und Aufwand, und steigert den Widerstandsbeiwert.

Bei der Betrachtung der $C_{W}-n$-Verl"aufe in Abh"angigkeit von den Impulskoeffizienten ist ersichtlich, dass diese mit steigendem $C_{\mu}$ degressiv sinken und eine Ausblasung unabh"angig von der Drehzahl einen gro\ss{}en ANteil an der Widerstandsreduzierung hat.

Bei den ovalen Wellen ist die Abh"angigkeit der $C_{W}$-Werte von den Impulskoeffizienten "ahnlich wie bei der Basiskonfiguration (\abb{fig:Cw/n bei Cmu 50}).
\begin{figure}[h]
	\centering
	\includegraphics[width=0.5\textwidth]{Cw_ueber_n_fuer_verschiedeneCmu_50.jpg}
	\caption{$C_{w}$ "uber $n$ bei verschiedenen $C_{\mu}$ f"ur die Konfiguration 50}
	\label{fig:Cw/n bei Cmu 50}
\end{figure}

Die zus"atzliche Rotation der Walzen ist aber nur bei bestimmten $C_{\mu}$ und Drehzahlen widerstandsreduzierend. Bei allen $C_{\mu}$ wird der $C_{W}$ bei Beginn der Rotation erh"oht, bis er wieder im Bereich der Abl"osefrequenz einbricht. Allerdings ist Sensibilit"at bei h"oheren $C_{\mu}$ in diesem Frequenzbereich sehr hoch, sodass der Widerstand deutlich erh"oht wird, wenn die optimale Drehzahl nicht getroffen wird.

In der \abb{fig:CwCwref/n Cmu RK} sind die mit dem jeweiligen $C_{ref}$ normierten $C_{W}$-Werte der Basiskonfiguration, bei verschiedenen $C_{\mu}$ zu sehen.
Als jeweiliges $C_{ref}$ dient hierbei der $C_{W}$-Wert des entsprechenden $C_{\mu}$s ohne Rotation.


\begin{figure}[h]
	\centering
	\begin{subfigure}[c]{0.45\textwidth}		
		\includegraphics[width=1\textwidth]{CwCw0_ueber_n_fuer_2Cmus_RK.jpg}
		%\subcaption{}
	\end{subfigure}
	\begin{subfigure}[c]{0.45\textwidth}
		\includegraphics[width=1\textwidth]{CwCw0_ueber_n_fuer_Cmu3und4_RK.jpg}
		%\subcaption{}
	\end{subfigure}
	\begin{subfigure}[c]{0.45\textwidth}
		\includegraphics[width=1\textwidth]{CwCw0_ueber_n_fuer_Cmu5_RK.jpg}
		%\subcaption{}
	\end{subfigure}
	\caption{$C_{W}/C_{Wref}$ "uber n bei verschiedenen $C_{\mu}$ f"ur die Basiskonfiguration}
	\label{fig:CwCwref/n Cmu RK}
\end{figure}

In der \abb{fig:CwCwref/n Cmu 50} sind die mit dem jeweiligen $C_{ref}$ normierten $C_{W}$-Werte der Konfiguration 50, bei verschiedenen $C_{\mu}$ zu sehen.
Als jeweiliges $C_{ref}$ dient hierbei der $C_{W}$-Wert des entsprechenden $C_{\mu}s$ ohne Rotation.

\begin{figure}[h]
	\centering
	\includegraphics[width=0.5\textwidth]{CwC0_ueber_n_fuer_Cmus_50.jpg}
	\caption{$C_{W}/C_{Wref}$ "uber n bei verschiedenen $C_{\mu}$ f"ur die Konfiguration 50}
	\label{fig:CwCwref/n Cmu 50}
\end{figure}

In den \abb{fig:PR} sind die Leistungsraten beider Konfigurationen bei verschiedenen Impulskoeffizienten und Drehzahlen ersichtlich.\\
Ein $PR= 1$ bedeutet, dass man so viel Energie mit der Widerstandsreduzierung erspart hat, wie man Energie f"ur dieselbe eingesetzt hat. In so einem Fall wird man sich in der Realit"at gegen diese widerstandsreduzierenden Ma\ss{}nahmen entscheiden, da diese nur erh"ohten Zeitaufwand, h"ohere Kosten und Reparaturanf"allifgkeit, ein gesteigertes Gewicht und weitere negative Nebenwirkungen zur Folge haben.

\begin{figure}[h]
	\centering
	\begin{subfigure}[c]{0.45\textwidth}		
		\includegraphics[width=1\textwidth]{PR_uber_n_fuer_Cmu_RK.jpg}
		\subcaption{$PR$ f"ur die Basiskonfiguration}
		\label{fig:PR RK}
	\end{subfigure}
	\begin{subfigure}[c]{0.45\textwidth}
		\includegraphics[width=1\textwidth]{PR_ueber_n_fuer_Cmu_50.jpg}
		\subcaption{$PR$ f"ur die Konfiguration 50}
		\label{fig:PR 50}
	\end{subfigure}
		\caption{$PR$ "uber $n$ bei verschiedenen $C_{\mu}$ f"ur beide Konfigurationen}
	\label{fig:PR}
\end{figure}

Es wird deutlich, dass die Konfiguration 50 bei keiner Drehzahl und keinem $C_{\mu}$ den Wert 1 erreicht, oder "uberschreitet. Der wichtigste zu nennende Grund daf"ur ist die gro\ss{}e Reibung zwischen der Teflonschicht und dem Ausblaseschlitz, die durch die Motoren "uberwunden werden muss. Diese aufzubringende Kraft ist unter anderem so gro\ss{}, da man gewisse Vorspannkr"afte aufbringen musste, um den vordefinierten Signalverlauf sicherzustellen.

Die Basiskonfiguration erreicht im Vergleich zu der gezahnten Konfiguration eine deutlich h"ohere Effektivit"at, auch wenn nicht immer vorteilhaft, da gerade nur bei zwei $C_{\mu}$s und einer Drehzahl bis ca. $1000$\,$min^{-1}$ Leistung erspart werden kann. Allerdings wird deutlich, dass die Bestwerte bei reiner Ausblasung erreicht werden. Die Widerstandsreduzierung der rotierenden Walzen verursacht einen gro\ss{}en energetischen Aufwand, sodass die Leistungsersparnis trotz einer Widerstandsreduzierung kleiner wird oder sogar in Leistungsverlust umgewandelt wird.
\chapter{Fazit}\label{s:fazit}


% Literaturverzeichnis
\newpage
\fancyhead[RE]{Literaturverzeichnis}
\fancyhead[LO]{Literaturverzeichnis}
\addcontentsline{toc}{chapter}{Literaturverzeichnis}
\bibliographystyle{plain}
%\bibliographystyle{bib_name_year}
\bibliography{lit_all} 


% Abbildungsverzeichnis
\newpage
\fancyhead[RE]{Abbildungsverzeichnis}
\fancyhead[LO]{Abbildungsverzeichnis}
{
   \renewcommand{\baselinestretch}{0.85}   %% Zeilenabstand / TOC
   \small\normalsize			   %% neuen Zeilenabstand aktivieren
   \addcontentsline{toc}{chapter}{Abbildungsverzeichnis}
   \listoffigures
}


% Tabellenverzeichnis
\newpage
\fancyhead[RE]{Tabellenverzeichnis}
\fancyhead[LO]{Tabellenverzeichnis}
{
   \renewcommand{\baselinestretch}{0.85}    %% Zeilenabstand / TOC
   \small\normalsize                        %% neuen Zeilenabstand aktivieren
   \addcontentsline{toc}{chapter}{Tabellenverzeichnis}
   \listoftables
}


% Anhang
\fancyhead[RE]{\nouppercase{\leftmark}}
\fancyhead[LO]{\rightmark}
\begin{appendix}
   \chapter{Technsiche Zeichnungen}\label{s:anh_TZ}


Text.


   \chapter{MATLAB-Code}
\label{c:Anhang B}

\section{Bestimmung der Druckverteilungen und des Widerstands}
\label{C_w-Code}
Text.


\end{appendix}

\end{document}
