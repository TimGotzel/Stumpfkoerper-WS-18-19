\chapter*{Zusammenfassung}\label{s:uebersicht}

In dieser Arbeit wird die m"ogliche Widerstandsreduktion an einem D-f"ormigen Stumpfk"orper durch eine periodische hybride Aktuationsmethode untersucht und mit der konstanten Aktuation auch in Hinsicht auf eine Nettoleistungsersparnis verglichen.
Die aktive Str"omungseeinflussung kombiniert dabei die Grenzschichtbeeinflussung durch sich drehende, gezahnte Zylinder mit der Ausblasung von Coand\^{a}-Jets an der Hinterkante des Modells.

Es kann eine Widerstandsreduktion durch die periodische Aktuation im Bereich der nat"urlichen Abl"osefrequenz des K"orpers beobachtet werden.

Allerdings werden die Ergebnisse und Leistungskoeffizienten durch einige sch"adliche Effekte in ihrer Aussagekraft beeintr"achtigt, sodass weitere Untersuchungen zu dem Thema vonn"oten sind.
