\chapter{Fazit}\label{s:fazit}
Im Rahmen dieser Projektarbeit wurde ein hybrider Aktuationsmechanismus auf eine m"ogliche Widerstandsreduktion an einem D-Stumpfk"orpermodell "uberpr"uft.
Der hybride Aktuationsmechanismus setzte sich dabei aus rotierenden Wellen und der Ausblasung von Coand\^{a}-Jets zusammen.
Im Fokus der Projektarbeit stand dabei die Frage, ob eine periodische Aktuation eine zus"atzliche Reduktion des Widerstands im Vergleich zur konstanten Aktuation bewirken kann und, ob dieser Mechanismus im Bezug auf die Leistungsbilanz sinnvoll eingesetzt werden k"onnte.

F"ur diese Zwecke wurden zwei "ahnliche, gezahnte Paare Walzen aus PTFE konstruiert und gefertigt, die f"ur eine eine periodische Ausblasung mit definiertem Signal und jeweils unterschiedlichen duty cyclen sorgen sollten. Ein Paar war wegen zu gro\ss{}er Fertigungsungenauigkeiten nicht f"ur die Versuche verwendbar.

Die Messdaten wurden "uber einen Druckmessrechen im LNB des ISM der TU Braunschweig bei verschiedenen Kombinationen der Ausblaseintensit"at und Walzenumdrehungen pro Minute gewonnen.

Im Anschluss wurden die Widerstandsbeiwerte  und verschiedene zugeh"orige Leistungskoeffizienten bestimmt und diese Daten hinsichtlich der Effektivit"at der Widerstandsreduktion und der Leistungsbilanz der einzelnen Konfigurationen, sowie m"oglichen Synergieeffekten ausgewertet.

Die Ergebnisse bez"uglich konstanter Aktuation von \textit{Bilges}, die eine Leistungsersparnis durch die Widerstandsreduktion bei einer begrenzten Zahl an Kombinationen der Aktuationsparameter im unteren Intensit"atsbereich festgestellt haben, konnten best"atigt werden. 

Es wurde zudem gezeigt, dass die periodische Aktuation im Bereich knapp oberhalb der nat"urlichen Abl"osefrequenz von 1938\,$\mathrm{min^{-1}}$ einen deutlichen positiven Einfluss auf den Widerstand des stumpfen K"orpers hat.

Allerdings zeigten die gefertigten gezahnten Wellen bei Rotation Abrieberscheinungen, die zus"atzlichen in einer deutlich geschm"alerten  Leistungsbilanz m"undeten.
So stieg der Leistungskoeffizient $C_{Power}$ der Motoren  im Durchschnitt auf das etwa Dreifache im Vergleich zur Basiskonfiguration.
Eine weitere Schwierigkeit, die im Zuge der periodischen Aktuation auftrat, war, die Phasengleichheit der rotierenden Wellen an Ober- und Unterseite sicherzustellen.

So kann die ausg"angliche Fragestellung durch die Projektarbeit nicht abschlie\ss{}end gekl"art werden. Vielmehr ist es notwendig diese Art der aktiven Str"omungsbeeinflussung in zuk"unftigen Arbeiten mit mehreren Modifikationen erneut zu testen. Denkbar w"aren beispielsweise eine andere Auswahl bzw. Anordnung der  Materialien die an der Paarung von Spaltlippe und Welle beteiligt sind. Dar"uber hinaus muss in Zukunft die Phasensynchronisierung auf mechanische oder elektronische Weise gew"ahrleistet werden.