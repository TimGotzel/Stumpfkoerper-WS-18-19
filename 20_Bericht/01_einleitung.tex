\chapter{Einleitung (NB)}\label{s:einleitung}
Im Rahmen dieser Arbeit werden Experimente an D-f"ormigen Stumpfk"orpern im Windkanal durchgef"uhrt. Stumpfk"orper haben einen gro\ss{}en Luftwiderstand, der haupts"achlich aus dem Druckwiderstand besteht. Es soll ein neuer Ansatz untersucht werden, der es erm"oglichen soll, den Widerstand des K"orpers zu verringern.\\
Diese Arbeit folgt auf die Masterarbeit \cite{Bilges.2018} von Bilges am Institut f"ur Str"omungsmechanik der Technischen Universit"at Braunschweig. Hier wurde eine Widerstandsreduktion durch die Kombination von zwei verschiedenen widerstandsreduzierenden Ans"atzen erreicht. Dabei wurde ein "uber eine Co\^{a}nda-Fl"ache geblasener Luftstrahl mit einer bewegten Oberfl"ache kombiniert. Das Stumpfk"orpermodell und der Aktuationsmechanismus sind f"ur diese Arbeit "ubernommen worden. Am Ende des Modells befinden sich zwei rotierende Walzen, die zus"atzlich mit einer Verzahnung versehen sind und sich drehen. "Uber diese Walzen wird impulsartig Druckluft in die Str"omung eingebracht. Hinter dem Modell werden die Druckverl"aufe gemessen, sodass der Widerstand des K"orpers bestimmt werden kann.\\
Ziel der Arbeit ist es heraus zu finden, ob eine gesteigerte Widerstandsreduktion mit dieser Art der Anregung festgestellt werden kann. Au\ss{}erdem soll im Bezug auf eine praktische Anwendung von bspw. LKW`s eine Energiebetrachtung durchgef"uhrt werden.\\
Da es sich bei diesem Dokument um eine Projektarbeit handelt, an der insgesamt f"unf Personen mitgewirkt haben, stehen hinter jeder Kapitel- bzw. Unterkapitel"uberschrift die Initialen des Autors. In \tab{tab:initialien} ist eine Aufschl"usselung der Initialien gegeben.

\begin{table}[h]
	\centering
	\begin{tabular}{lr}
		\toprule
		Name & Initialen\\
		\midrule
		Nora M. Bierwagen & NB\\
		Tim Gotzel & TG\\
		Amiriman Kianfar & AK\\
		Kebria Kiani & KK\\
		Florian Timm & FT\\
		\bottomrule
	\end{tabular}
	\caption{Initialen der beteiligten Personen}
	\label{tab:initialien}
\end{table}
